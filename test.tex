\documentclass[12pt,a4paper]{report}
\usepackage[utf8]{inputenc}
\usepackage[french]{babel}
\usepackage{amsmath}
\usepackage{geometry}
\geometry{margin=2.5cm}

\title{Test d'Analyse Multi-Agent IA}
\author{Votre Nom}
\date{\today}

\begin{document}

\maketitle
\tableofcontents
\newpage

\chapter{Introduction}

Ce document est un test simple pour vérifier le fonctionnement de l'analyseur multi-agent IA. 
L'objectif est de démontrer que le système peut analyser correctement un document LaTeX académique.

Les mathématiques sont au cœur de nombreuses disciplines scientifiques. Par exemple, 
la célèbre équation d'Einstein s'écrit :
\begin{equation}
E = mc^2
\end{equation}

Cette équation établit l'équivalence entre masse et énergie, un principe fondamental 
en physique moderne.

\section{Contexte historique}

Albert Einstein a publié cette équation en 1905 dans le cadre de sa théorie de la 
relativité restreinte. Cette découverte a révolutionné notre compréhension de l'univers.

\section{Objectifs du document}

Ce document test a pour objectifs de :
\begin{itemize}
\item Vérifier l'extraction correcte des chapitres et sections
\item Tester l'analyse scientifique des équations mathématiques
\item Évaluer la qualité de l'analyse stylistique
\item Générer un rapport d'analyse structuré
\end{itemize}

\chapter{Développement théorique}

\section{Fondements mathématiques}

Les mathématiques fournissent un langage universel pour décrire les phénomènes naturels. 
Considérons l'équation différentielle suivante :
\begin{equation}
\frac{dy}{dx} = ky
\end{equation}

Cette équation modélise de nombreux processus naturels, de la décroissance radioactive 
à la croissance des populations.

\section{Applications pratiques}

Les équations mathématiques trouvent des applications dans de nombreux domaines :
\begin{enumerate}
\item Physique : mécanique quantique, thermodynamique
\item Ingénierie : conception de structures, systèmes de contrôle
\item Économie : modélisation financière, optimisation
\item Biologie : dynamique des populations, épidémiologie
\end{enumerate}

\subsection{Exemple en physique}

L'équation de Schrödinger dépendante du temps s'écrit :
\begin{equation}
i\hbar\frac{\partial\psi}{\partial t} = \hat{H}\psi
\end{equation}

Cette équation est fondamentale en mécanique quantique et décrit l'évolution 
temporelle d'un système quantique.

\subsection{Exemple en ingénierie}

Le théorème de Pythagore, bien qu'ancien, reste essentiel en ingénierie :
\begin{equation}
a^2 + b^2 = c^2
\end{equation}

Il permet de calculer des distances et d'optimiser des structures triangulaires.

\chapter{Méthodologie}

\section{Approche générale}

Notre approche se base sur une analyse systématique des concepts mathématiques. 
Nous procédons en trois étapes :

\begin{enumerate}
\item Identification des équations clés
\item Analyse de leur cohérence théorique
\item Validation des applications pratiques
\end{enumerate}

\section{Outils d'analyse}

L'analyse utilise plusieurs outils mathématiques modernes, notamment :
\begin{itemize}
\item Calcul différentiel et intégral
\item Algèbre linéaire
\item Théorie des probabilités
\item Méthodes numériques
\end{itemize}

\chapter{Résultats}

\section{Observations principales}

Nos analyses révèlent plusieurs points importants concernant l'application 
des mathématiques dans les sciences. Les équations étudiées montrent une 
grande cohérence théorique et pratique.

\section{Validation des hypothèses}

Les hypothèses initiales ont été largement validées par nos analyses. 
La rigueur mathématique se confirme dans tous les exemples étudiés.

\chapter{Conclusion}

\section{Synthèse des résultats}

Ce document test a permis de vérifier le fonctionnement de l'analyseur multi-agent IA. 
Les résultats montrent que :

\begin{enumerate}
\item L'extraction des chapitres fonctionne correctement
\item Les équations mathématiques sont bien identifiées
\item L'analyse peut être effectuée de manière structurée
\item Un rapport cohérent peut être généré
\end{enumerate}

\section{Perspectives}

L'analyseur multi-agent IA ouvre de nouvelles perspectives pour l'amélioration 
de documents académiques. Il permet une analyse approfondie et objective, 
tout en proposant des pistes d'amélioration concrètes.

\section{Recommandations}

Pour une utilisation optimale de l'analyseur, nous recommandons de :
\begin{itemize}
\item Commencer par des documents de taille réduite
\item Comparer les résultats de différents modèles IA
\item Intégrer progressivement les suggestions d'amélioration
\item Maintenir une révision humaine finale
\end{itemize}

\chapter*{Remerciements}

Nous remercions les développeurs des API Claude (Anthropic), Gemini (Google) 
et GPT-4 (OpenAI) qui rendent possible cette analyse automatisée de documents 
académiques.

\appendix

\chapter{Annexe technique}

\section{Spécifications}

Ce document test contient :
\begin{itemize}
\item 5 chapitres principaux
\item 12 sections et sous-sections
\item 6 équations mathématiques
\item Environ 1000 mots
\end{itemize}

\section{Commandes utilisées}

Les principales commandes LaTeX utilisées sont :
\begin{verbatim}
\chapter{...}
\section{...}
\subsection{...}
\begin{equation}...\end{equation}
\begin{itemize}...\end{itemize}
\end{verbatim}

\end{document}